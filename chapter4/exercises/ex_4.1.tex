% === Exercise 4.1 ===
\begin{Exercise}
	\begin{answer}
		No.
	\end{answer}
	\begin{proof}
		Consider the function
		$$
		f(x) = \begin{cases}
		0 & \mbox{ for } x=0 \\
		1 & \mbox{ otherwise}
		\end{cases}.
		$$
		This function satisfies
		$$
		\lim_{h\to 0}\left[f(x+h)-f(x-h)\right] = 0
		$$
		for all $x\in\mathbb{R}^1$. However we claim $f$ is discontinuous at $x=0$.
		
		Given $\epsilon < 1$, for any $\delta >0$, we know for $x\in\mathbb{R}^1$, 
		$$
		0<|x-0|<\delta \implies |f(x)-f(0)| = 1 \geq \epsilon.
		$$
		So we have given a counter-example.
	\end{proof}
\end{Exercise}