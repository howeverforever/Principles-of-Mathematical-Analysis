% === Exercise 5.5 ===
\begin{Exercise}
	\begin{proof}
		Since $f'(x)\to 0$ as $x\to\infty$, then given $\epsilon>0$, there exists $N>0$ such that
		$$
		x\geq N \implies |f'(x)| < \epsilon.
		$$
		By the Mean Value Theorem, we consider
		$$
		g(x) = f(x+1)-f(x)
		= \frac{f(x+1)-f(x)}{(x+1)-x}
		= f'(c)
		$$
		where $c\in(x,x+1)$. 
		We pick $x \geq N$, then this means $c>x\geq N$. So
		$$
		|g(x)| = |f'(c)| < \epsilon.
		$$
		It follows that $g(x)\to 0$ as $x\to\infty$.
	\end{proof}
\end{Exercise}