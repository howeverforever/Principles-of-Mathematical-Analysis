% === Exercise 7.5 ===
\begin{Exercise}
	\begin{itemize}
		\item \textbf{Show that $\{f_n\}$ converges to a continuous function, but not uniformly.}
	\end{itemize}
	\begin{proof}
		Considering for $x\leq 0$ and $x > 1$, we observe $f_n(x) = 0$ for all $n\in\mathbb{N}$.
		Then $f(x) = \lim_{n\to\infty}f_n(x) = 0$ trivially.
		For any other value $x$, we just choose $N$ such that $N>\frac{1}{x}$ to make 
		$$
		n\geq N \implies x > \frac{1}{n}.
		$$
		Then $f(x) = \lim_{n\to\infty}f_n(x) = 0$, too.
		So far, we observe $\{f_n\}$ converges to $f(x) = 0$ which is continuous.
		
		To show that $f_n\to f$ not uniformly, let $0<\epsilon<1$ be given.
		We consider any $n$, then there always exists a point $x = \frac{2}{2n+1}$ such that
		$$
		f_n(x) 
		= \sin^2 \frac{\pi}{x} 
		= \sin^2 \left(n\pi + \frac{\pi}{2} \right)
		= 1.
		$$
		It follows that
		$$
		|f_n(x) - f(x)| = 1 \geq \epsilon.
		$$
		Hence $f_n\to f$ not uniformly.
	\end{proof}
	
	\begin{itemize}
		\item \textbf{Show that absolute convergence does not imply uniform convergence.}
	\end{itemize}
	\begin{proof}
		Notice that $|f_n| = f_n$ for all $x$.
		Since $\{f_n\}$ converges to $f$, then $\{|f_n|\}$ also converges to $f$.
		Hence $\sum f_n$ converges absolutely.
		However, $\{f_n\}$ does not converge uniformly from the previous argument.
	\end{proof}
\end{Exercise}