% === Exercise 7.20 ===
\begin{Exercise}
	\begin{proof}
		Let $P(x)$ be an arbitrary polynomial such that
		$$
		P(x) = a_n x^n + \cdots + a_1 x + a_0.
		$$
		We observe
		$$
		\int_{0}^{1} f(x) P(x)\,dx
		= a_n \int_{0}^{1} f(x)x^n\,dx + \cdots + a_1 \int_{0}^{1} f(x)x\,dx + a_0 \int_{0}^{1} f(x)\,dx
		= 0.
		$$
		By the Stone-Weierstrass Theorem, we can pick a sequence of polynomials $P_n(x)$ such that
		$$
		\lim_{n\to\infty} P_n(x) = f(x)
		$$
		uniformly on $[0,1]$.
		By Theorem 7.16, we know
		$$
		0
		= \lim_{n\to\infty}\int_{0}^{1}f(x) P_n(x)\,dx
%		↓ Here is maybe incredible.
		= \int_{0}^{1} \lim_{n\to\infty} f(x) P_n(x)\,dx
		= \int_{0}^{1} f(x) \lim_{n\to\infty} P_n(x)\,dx
		= \int_{0}^{1} f^2(x)\,dx
		$$
		Since $f$ is continuous on $[0,1]$, and so is $f^2$.
		Moreover, $f^2(x) \geq 0$ for all $x\in[0,1]$.
		By Exercise 6.2, we know $f^2(x) = 0$ on $[0,1]$.
		It follows that $f(x) = 0$ on $[0,1]$ immediately.
	\end{proof}
\end{Exercise}